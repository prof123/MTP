% Created 2011-09-05 Mon 14:02
\documentclass[11pt]{article}
\usepackage[latin1]{inputenc}
\usepackage[T1]{fontenc}
\usepackage{fixltx2e}
\usepackage{graphicx}
\usepackage{longtable}
\usepackage{float}
\usepackage{wrapfig}
\usepackage{soul}
\usepackage{textcomp}
\usepackage{marvosym}
\usepackage{wasysym}
\usepackage{latexsym}
\usepackage{amssymb}
\usepackage{hyperref}
\tolerance=1000
\providecommand{\alert}[1]{\textbf{#1}}
\begin{document}



\title{experiments1}
\author{prateek sharma}
\date{05 September 2011}
\maketitle

\setcounter{tocdepth}{3}
\tableofcontents
\vspace*{1cm}

\section{Page Sharing}
\label{sec-1}

Survey of everything. Some history?
\subsection{Scanning vs Disk based sharing}
\label{sec-1_1}

2 competing approaches.
   (2)
\section{KSM}
\label{sec-2}

Diagram of operations, history, and some implementation details
Focus on generic nature
\subsection{Exp 1: KSM effectiveness}
\label{sec-2_1}

Small experiment which shows that KSM shares large \% of same pages. Done earlier in fingerprinting project.
Expected result : > 90\% sharing of KSM, so good enough.
Reason : establish some ground truths: KSM \underline{works} .
Setup : Random workload (doesnt matter) and take fingerprint and see KSM shar\% . Run with 1,2,3 VMs.
   (1)
\section{Analysis of shared pages}
\label{sec-3}
\subsection{Exp 2: Pages shared by flag type}
\label{sec-3_1}

Run some benchmarks (static VMs just booted up ; Kernbench ; HTTP-perf) and see what kinds of pages are shared by flag type.
Reason: establish some ground truths : sharing is \underline{feasible} . Also answer : what \underline{kinds} of pages are shared?
Setup: (1,2,3,5) VMs with different OS running same benchmarks. (diff VMs ; same kernel ; same /var/www) 
\subsection{Exp 2.1 : KSM with no pagecache pages}
\label{sec-3_2}

Run KSM but skip all guest pagecache pages.
\subsection{Exp 2.2: Ftrace Overhead for KSM}
\label{sec-3_3}

Since we are recording all KSM events (just for few experiments only) whats the overhead of that?
1GB/minute data collected.
\subsection{Exp 3: Page sharing over time}
\label{sec-3_4}

Run some benchmarks and record pages shared over the duration of benchmark. Also record \textbf{KSM overhead}.
Reason: Show that KSM overhead is significant enough, thus implying the need for some optimizations.
Setup: (2) VMs running benchmarks. KSM being profiled using perf.

(1)
\section{Lookahead optimization}
\label{sec-4}

  
\subsection{Exp 4: Lookahead success}
\label{sec-4_1}

Run benchmarks on VMs (1,2,3) to on and record lookahead successes. Also record \textbf{KSM overhead}
\textbf{Compare vanilla KSM overhead with lookahead-optimization}
\subsection{Exp 5: Substrings in shared-map.}
\label{sec-4_2}

Record consecutive pages being shared in some benchmarks.
Reason : justify why lookahead works.
Setup: tracedump analysis simple python script 

   (3)
\section{Problem of double-caching}
\label{sec-5}
\subsection{Exp 6: Memory savings with exclusive caches}
\label{sec-5_1}

How many pages are there in both places?
Setup : Benchmarks on VMs.
\subsection{Exp 7: Overhead of ksm-exclusive-cache}
\label{sec-5_2}

Run benchmarks on VMs to record KSM overhead (with ex cache)
Reason : scanning vast host page cache could be significant overhead.Also savings might help.

Some caching theory references.
   (1)
\section{Qualitative survey of dynamic memory management for VMs.}
\label{sec-6}
\subsection{Exp 8: Memory mountains}
\label{sec-6_1}

Look at this problem like L1/2 cache , and build mem-mountains in these cases:
(normal ; no guest cache ; no host cache ; swap as ramdisk )
Setup : could use IOZone or randal bryant's simple program.
Reason : Demonstrate the latencies/throughput of various caches. 
\textbf{This depends on lots of factors like IO schedulers, FS, virtual disk layout etc. Do for any one, for now.}

\end{document}